\documentclass[a4paper,UKenglish,cleveref, autoref]{templates/lipics-v2019}
\usepackage{csquotes}
\usepackage{url}
%This is a template for producing LIPIcs articles.
%See lipics-manual.pdf for further information.
%for A4 paper format use option "a4paper", for US-letter use option "letterpaper"
%for british hyphenation rules use option "UKenglish", for american hyphenation rules use option "USenglish"
%for section-numbered lemmas etc., use "numberwithinsect"
%for enabling cleveref support, use "cleveref"
%for enabling cleveref support, use "autoref"

\newcommand{\sls}{Safer Language Subset}
\newcommand{\slss}{Safer Language Subsets}

\newcommand{\sqs}{Software-Qualitätsstandard}
\newcommand{\sqss}{Software-Qualitätsstandards}

\newcommand{\misra}{MISRA-C}

\newcommand{\p}[1]{S. #1}

\graphicspath{{./graphics/}}%helpful if your graphic files are in another directory

\bibliographystyle{plainurl}% the mandatory bibstyle

\title{M10: \sqss}

\titlerunning{M10: \sqss}%optional, please use if title is longer than one line

\author{Alexander Linder}{Karlsruhe Institute of Technology, Germany \and \url{https://kit.edu} }{alexander.linder@student.kit.edu}{}{}
%mandatory, please use full name; first two parameters are mandatory, other parameters can be empty.

\authorrunning{A. Linder}
%mandatory. First: Use abbreviated first/middle names. Second (only in severe cases): Use first author plus 'et al.'

\Copyright{Alexander Linder}
%%mandatory, please use full first names. LIPIcs license is "CC-BY";  http://creativecommons.org/licenses/by/3.0/

\ccsdesc[100]{General and reference}
%mandatory: Please choose ACM 2012 classifications from https://dl.acm.org/ccs/ccs_flat.cfm

\keywords{\misra, \sqss}
%mandatory; please add comma-separated list of keywords

\category{}
%optional, e.g. invited paper

\relatedversion{}
%optional, e.g. full version hosted on arXiv, HAL, or other respository/website

\supplement{}
%optional, e.g. related research data, source code, ... hosted on a repository like zenodo, figshare, GitHub, ...

\funding{}
%optional, to capture a funding statement, which applies to all authors. Please enter author specific funding statements as fifth argument of the \author macro.

\acknowledgements{}
%optional

\nolinenumbers %uncomment to disable line numbering

\hideLIPIcs  %uncomment to remove references to LIPIcs series (logo, DOI, ...), e.g. when preparing a pre-final version to be uploaded to arXiv or another public repository

%Editor-only macros:: begin (do not touch as author)%%%%%%%%%%%%%%%%%%%%%%%%%%%%%%%%%%
\EventEditors{}
\EventNoEds{0}
\EventLongTitle{Proseminar Werkzeuge und Methoden der Software-Analyse}
\EventShortTitle{Proseminar}
\EventAcronym{Proseminar}
\EventYear{2019}
\EventDate{WS19/20}
\EventLocation{Karlsruhe, Deutschland}
\EventLogo{}
\SeriesVolume{}
\ArticleNo{M10}
%%%%%%%%%%%%%%%%%%%%%%%%%%%%%%%%%%%%%%%%%%%%%%%%%%%%%%

\begin{document}

\maketitle

%TODO mandatory: add short abstract of the document
\begin{abstract}
    Diese Arbeit gibt einen Überblick über \sqss\ im Allgemeinen sowie unter besonderer Betrachtung eines weitverbreiteten Beispiels, \misra\ 2004.\\
    Es wird die hinter \sqss\ stehende Motivation dargebracht, sowie betrachtet wie sich Regeln innerhalb eines Standards sinnvoll klassifizieren lassen.
    Dies führt zum Begriff des \slss\ nach Hatton~\cite{hatton2004safer}, welches formale Forderungen an einen guten Regelsatz aufstellt.
    Weiterhin wird überprüft, inwiefern \misra\ diese Definition erfüllt.\\
    Final sollen Gütekriterien für \sqss\ aufgestellt werden und die Güte des \misra\ Standards unter eben diesen Kriterien bewertet werden.
\end{abstract}

\section{Einleitung}
\label{sec:einleitung}

\section{\misra}
\label{sec:misra-c}
\misra\ ist ein weitverbreiteter \sqs, der von der \textbf{M}otor \textbf{I}ndustry \textbf{S}oftware \textbf{R}eliability \textbf{A}ssociation (MISRA) entwickelt wurde.
Die erste Version von \misra\ wurde 1998 veröffentlicht, mit insgesamt zwei großen Revisionen einmal 2004 sowie 2012,
wobei sich diese Arbeit auf die Standards von 1998 und 2004 beschränkt, da einerseits der Standard von 2004 nach wie vor
weite Verbreitung findet und sich andererseits die Forschung auf diese beiden Versionen konzentriert.\\
Bei \misra\ handelt es sich um den de facto Standard zur Programmierung eingebetteter Systeme in der Mehrheit der sicherheitsrelevanten Industrien in der Programmiersprache C\@.\cite{misra-website}
Sowohl \misra\ 1998 als auch 2004 nehmen Bezug nehmen Bezug auf C89/90 und nicht auf den neueren C-Standard C99 (ISO/IEC9899:1999), was dazu führt, dass neuere Sprachkonstrukte in
\misra-konformem Code nicht erlaubt sind, auch wenn die verwendeten Compiler sie unterstützen.

\section{Regelklassifizierung}
\label{sec:regelklassifizierung}

\subsection{Typ-A Regeln}
\label{subsec:typ-a-regeln}

\subsection{Typ-B.1 Regeln}
\label{subsec:typ-b-1-regeln}

\subsection{Typ-B.2 Regeln}
\label{subsec:typ-b-2-regeln}

\section{Safer Language Subsets}
\label{sec:safer-language-subsets}

\section{Bewertung}
\label{sec:bewertung}

\section{Konklusion}
\label{sec:konklusion}

%%
%% Bibliography
%%

%% Please use bibtex,

\bibliography{prosem}

\end{document}
