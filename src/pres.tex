\documentclass{beamer}

\usepackage{listings}
\usepackage[T1]{fontenc}
\usepackage{graphicx}
\usepackage[utf8]{inputenc}
\usepackage{xcolor}
\usepackage{amsmath,amssymb}
\usepackage{mathtools}
\usepackage[linesnumbered,lined,boxed,commentsnumbered,ruled,vlined]{algorithm2e}
\usepackage{csquotes}
\usepackage{templates/beamerthemekit}

\newcommand{\sls}{Safer Language Subset}
\newcommand{\slss}{Safer Language Subsets}

\newcommand{\sqs}{Software-Qualitätsstandard}
\newcommand{\sqss}{Software-Qualitätsstandards}

\newcommand{\misra}{MISRA-C}

\newcommand{\p}[1]{S. #1}

\lstset{language=C}

\titleimage{Titelbild}

\titlelogo{empty}

\graphicspath{{./graphics/}}%helpful if your graphic files are in another directory

\title{M10: \sqss}
\author[Linder]{Alexander Linder}
\date{} %TODO add date of presentation
\institute{FAKULTÄT FÜR INFORMATIK}
%\titlegraphic{\includegraphics[scale=0.1]{title.png}} % TODO add title image

\begin{document}
    \setbeamercovered{invisible}
    % change the following line to "ngerman" for German style date and logos
    \selectlanguage{ngerman}

    \begin{frame}
        \maketitle
    \end{frame}

    \begin{frame}
        \tableofcontents
    \end{frame}

    \section{Motivation}
    \label{sec:motivation}
    \begin{frame}{Motivation von \sqss}
        Was ist unsere Motivation?\\
        Warum wollen wir Standards?\\
        Was sind Vorteile dieser Standards?
    \end{frame}

    \section{\misra}
    \label{sec:misra-c}
    \begin{frame}{Entwicklung}
        %Entwicklung und Evolution des Standards
        Entwicklung und Evolution des Standards
    \end{frame}

    \begin{frame}{Über \misra}
        Inhalt, beteiligte Unternehmen/Branchen
    \end{frame}

    \begin{frame}{Regelsatz}
        %Übersicht des Regelsatzes
        Übersicht des Regelsatzes von \misra\
    \end{frame}

    \begin{frame}{Beispielregel}
        (möglichst kompaktes) Beispiel einer Regel aus dem \misra\ Regelsatz
    \end{frame}

    \section{Regelklassifizierung}
    \label{sec:regelklassifizierung}
    \begin{frame}{Motivation}
        \enquote{Warum} klassifizieren wir Regeln?
    \end{frame}

    \subsection{Typ A Regeln}
    \label{subsec:typ-a-regeln}
    \begin{frame}{\enquote{Style}-Regeln}
        \begin{block}{Definition}

        \end{block}
        \begin{exampleblock}{Beispiel}
            \lstinputlisting{graphics/typa.c}
        \end{exampleblock}
    \end{frame}

    \subsection{Typ B.1 Regeln}
    \label{subsec:typ-b-1-regeln}
    \begin{frame}{Subjektive Regeln}
        \begin{block}{Definition}

        \end{block}
        \begin{exampleblock}{Beispiel}
            \lstinputlisting{graphics/typb1.c}
        \end{exampleblock}
    \end{frame}

    \subsection{Typ B.2 Regeln}
    \label{subsec:typ-b-2-regeln}
    \begin{frame}{Evidenzbasierte Regeln}
        \begin{block}{Definition}

        \end{block}
        \begin{exampleblock}{Beispiel}
            \lstinputlisting{graphics/typb2.c}
        \end{exampleblock}
    \end{frame}

    \section{Safer Language Subsets}
    \label{sec:safer-language-subsets}
    \begin{frame}{Definition}
        Typ B.2 Regeln als verpflichtende Regeln\\
        Typ B.1 Regeln als empfohlene Regeln\\
        Keinerlei Typ A Regeln\\
    \end{frame}

    \begin{frame}{Warum genau \textit{diese} Definition?}
        Begründung voriger Definition
    \end{frame}

    \begin{frame}{Spezifikation}
        Wie spezifizieren wir solch ein \sls \\
        Welche Eigenschaften sollten die Regeln erfüllen und wie schreiben wir sie am besten nieder?
    \end{frame}

    \begin{frame}{Probleme}
        Probleme die bei der Spezifikation auftreten
    \end{frame}

    \section{Bewertung}
    \label{sec:bewertung}
    \begin{frame}{Auswertung des Regelnutzen}
        %Graphik hier
        Graphik hier
    \end{frame}

    \begin{frame}{Fazit}

    \end{frame}

\end{document}